\section{JML}
JML \cite{Patrice:06, Lilian:04} is a state-based specification languages. 
The language constructs for specifying an object-oriented module
include assertions, pre and postconditions, invariants, frame properties,
datagroups, and ghost and model fields
\begin{itemize}
  \item invariant: $\textbf{@ protected/public/private invariant}$ denotes
  object invariant;
  \item precondition: $\textbf{@requires}$
  \item postcondition: $\textbf{@ensures}$
  \item assignable: $\textbf{@assignable}$ specify which parts of the system
  state may change as the result of the method execution;
  \item frame properties: assignable clause specify the frame property.
  Any location outside the frame property is guaranteed to have the same value
  after the method has executed;
  \item datagroups: $\textbf{@ public model JMLDataGroup x}$, which abstract
  away from private implementation details in frame properties and provides flexibility in
  specifications; use $\textbf{@in x}$ to specify which variables are abstracted
  by abstract variable $x$;
  \item model field: $\textbf{@public model x}$ and $\textbf{@ private
  represents x = Exp(variables);}$ A model field is a specification-only field
  that pro- vides an abstraction of (part of) the concrete state of an object;
  \item ghost: $\textbf{@public ghost x}$ and $\textbf{@private/public/protected
  invariant x == Exp}$, where ghost fields are specification-only fields. While
  a model field provides an abstraction of the existing state, a ghost field can
  provide some additional state, which may—or may not—be related to the existing state.
    \begin{itemize}
      \item $\textbf{@set x = v}$: statement to update ghost variables, while
      model field will update automatically;
      \item A ghost field extends the state of an object, whereas a model field
      is an abstraction of the existing state of an object
    \end{itemize}
  \item other constructs: 
    \begin{itemize}
      \item $\textbf{specification inheritance}$: Class B extends A {\ldots},
      the specification for inherited action in B should be also satisfied by A;
      \item $\textbf{postconditions}$: include both normal postconditions and
      exceptional postconditions, which is expressed by means of
      $\textbf{signals}$ clauses, that must hold when a method terminates with
      an exception
      \item $\textbf{pure function}$: has no side effects, e.g. value reading
      function $\mathit{getTime()}$, which can be used in specification;
    \end{itemize}
\end{itemize}